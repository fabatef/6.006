%
% 6.006 problem set 1 solutions template
%
\documentclass[12pt,twoside]{article}

\usepackage{amsmath}
\usepackage{color}

% Cross-references for handout numbers.

\newcommand{\name}{}

\usepackage{latexsym}
%\usepackage{bbm}
\usepackage{times,url}
\usepackage{clrscode}

\newcommand{\mitst}[1]{\begin{description}
\item[MIT students:] #1
\end{description}}
\newcommand{\smast}[1]{\begin{description}
\item[SMA students:] #1
\end{description}}

\newcommand{\profs}{Professors Constantinos Daskalakis, Piotr Indyk, and Steve Smith}
\newcommand{\subj}{6.006}

\newlength{\toppush}
\setlength{\toppush}{2\headheight}
\addtolength{\toppush}{\headsep}

\newcommand{\htitle}[2]{\noindent\vspace*{-\toppush}\newline\parbox{6.5in}
{\textit{Introduction to Algorithms: 6.006}\hfill\name\newline
Massachusetts Institute of Technology \hfill #2\newline
\profs\hfill #1 \vspace*{-.5ex}\newline
\mbox{}\hrulefill\mbox{}}\vspace*{1ex}\mbox{}\newline
\begin{center}{\Large\bf #1}\end{center}}

\newcommand{\handout}[2]{\thispagestyle{empty}
 \markboth{#1}{#1}
 \pagestyle{myheadings}\htitle{#1}{#2}}

\newcommand{\htitlewithouttitle}[2]{\noindent\vspace*{-\toppush}\newline\parbox{6.5in}
{\textit{Introduction to Algorithms}\hfill#2\newline
Massachusetts Institute of Technology \hfill 6.006\newline
%Singapore-MIT Alliance \hfill SMA5503\newline
\profs\hfill Handout #1\vspace*{-.5ex}\newline
\mbox{}\hrulefill\mbox{}}\vspace*{1ex}\mbox{}\newline}

\newcommand{\handoutwithouttitle}[2]{\thispagestyle{empty}
 \markboth{Handout \protect\ref{#1}}{Handout \protect\ref{#1}}
 \pagestyle{myheadings}\htitlewithouttitle{\protect\ref{#1}}{#2}}

\newcommand{\exam}[2]{% parameters: exam name, date
 \thispagestyle{empty}
 \markboth{\subj\ #1\hspace{1in}Name\hrulefill\ \ }%
          {\subj\ #1\hspace{1in}Name\hrulefill\ \ }
 \pagestyle{myheadings}\examtitle{#1}{#2}
 \renewcommand{\theproblem}{Problem \arabic{problemnum}}
}
\newcommand{\examsolutions}[3]{% parameters: handout, exam name, date
 \thispagestyle{empty}
 \markboth{Handout \protect\ref{#1}: #2}{Handout \protect\ref{#1}: #2}
% \pagestyle{myheadings}\htitle{\protect\ref{#1}}{#2}{#3}
 \pagestyle{myheadings}\examsolutionstitle{\protect\ref{#1}} {#2}{#3}
 \renewcommand{\theproblem}{Problem \arabic{problemnum}}
}
\newcommand{\examsolutionstitle}[3]{\noindent\vspace*{-\toppush}\newline\parbox{6.5in}
{\textit{Introduction to Algorithms}\hfill#3\newline
Massachusetts Institute of Technology \hfill 6.006\newline
%Singapore-MIT Alliance \hfill SMA5503\newline
\profs\hfill Handout #1\vspace*{-.5ex}\newline
\mbox{}\hrulefill\mbox{}}\vspace*{1ex}\mbox{}\newline
\begin{center}{\Large\bf #2}\end{center}}

\newcommand{\takehomeexam}[2]{% parameters: exam name, date
 \thispagestyle{empty}
 \markboth{\subj\ #1\hfill}{\subj\ #1\hfill}
 \pagestyle{myheadings}\examtitle{#1}{#2}
 \renewcommand{\theproblem}{Problem \arabic{problemnum}}
}

\makeatletter
\newcommand{\exambooklet}[2]{% parameters: exam name, date
 \thispagestyle{empty}
 \markboth{\subj\ #1}{\subj\ #1}
 \pagestyle{myheadings}\examtitle{#1}{#2}
 \renewcommand{\theproblem}{Problem \arabic{problemnum}}
 \renewcommand{\problem}{\newpage
 \item \let\@currentlabel=\theproblem
 \markboth{\subj\ #1, \theproblem}{\subj\ #1, \theproblem}}
}
\makeatother


\newcommand{\examtitle}[2]{\noindent\vspace*{-\toppush}\newline\parbox{6.5in}
{\textit{Introduction to Algorithms}\hfill#2\newline
Massachusetts Institute of Technology \hfill 6.006 Spring 2014\newline
%Singapore-MIT Alliance \hfill SMA5503\newline
\profs\hfill #1\vspace*{-.5ex}\newline
\mbox{}\hrulefill\mbox{}}\vspace*{1ex}\mbox{}\newline
\begin{center}{\Large\bf #1}\end{center}}

\newcommand{\grader}[1]{\hspace{1cm}\textsf{\textbf{#1}}\hspace{1cm}}

\newcommand{\points}[1]{[#1 points]\ }
\newcommand{\parts}[1]
{
  \ifnum#1=1
  (1 part)
  \else
  (#1 parts)
  \fi
  \
}

\newcommand{\bparts}{\begin{problemparts}}
\newcommand{\eparts}{\end{problemparts}}
\newcommand{\ppart}{\problempart}

%\newcommand{\lg} {lg\ }

\setlength{\oddsidemargin}{0pt}
\setlength{\evensidemargin}{0pt}
\setlength{\textwidth}{6.5in}
\setlength{\topmargin}{0in}
\setlength{\textheight}{8.5in}


\newcommand{\Spawn}{{\bf spawn} }
\newcommand{\Sync}{{\bf sync}}

\renewcommand{\cases}[1]{\left\{ \begin{array}{ll}#1\end{array}\right.}
\newcommand{\cif}[1]{\mbox{if $#1$}}
\newcommand{\cwhen}[1]{\mbox{when $#1$}}

\newcounter{problemnum}
\newcommand{\theproblem}{Problem \theproblemsetnum-\arabic{problemnum}}
\newenvironment{problems}{
        \begin{list}{{\bf \theproblem. \hspace*{0.5em}}}
        {\setlength{\leftmargin}{0em}
         \setlength{\rightmargin}{0em}
         \setlength{\labelwidth}{0em}
         \setlength{\labelsep}{0em}
         \usecounter{problemnum}}}{\end{list}}
\makeatletter
\newcommand{\problem}[1][{}]{\item \let\@currentlabel=\theproblem \textbf{#1}}
\makeatother

\newcounter{problempartnum}[problemnum]
\newenvironment{problemparts}{
        \begin{list}{{\bf (\alph{problempartnum})}}
        {\setlength{\leftmargin}{2.5em}
         \setlength{\rightmargin}{2.5em}
         \setlength{\labelsep}{0.5em}}}{\end{list}}
\newcommand{\problempart}{\addtocounter{problempartnum}{1}\item}

\newenvironment{truefalseproblemparts}{
        \begin{list}{{\bf (\alph{problempartnum})\ \ \ T\ \ F\hfil}}
        {\setlength{\leftmargin}{4.5em}
         \setlength{\rightmargin}{2.5em}
         \setlength{\labelsep}{0.5em}
         \setlength{\labelwidth}{4.5em}}}{\end{list}}

\newcounter{exercisenum}
\newcommand{\theexercise}{Exercise \theproblemsetnum-\arabic{exercisenum}}
\newenvironment{exercises}{
        \begin{list}{{\bf \theexercise. \hspace*{0.5em}}}
        {\setlength{\leftmargin}{0em}
         \setlength{\rightmargin}{0em}
         \setlength{\labelwidth}{0em}
         \setlength{\labelsep}{0em}
        \usecounter{exercisenum}}}{\end{list}}
\makeatletter
\newcommand{\exercise}{\item \let\@currentlabel=\theexercise}
\makeatother

\newcounter{exercisepartnum}[exercisenum]
%\newcommand{\problem}[1]{\medskip\mbox{}\newline\noindent{\bf Problem #1.}\hspace*{1em}}
%\newcommand{\exercise}[1]{\medskip\mbox{}\newline\noindent{\bf Exercise #1.}\hspace*{1em}}

\newenvironment{exerciseparts}{
        \begin{list}{{\bf (\alph{exercisepartnum})}}
        {\setlength{\leftmargin}{2.5em}
         \setlength{\rightmargin}{2.5em}
         \setlength{\labelsep}{0.5em}}}{\end{list}}
\newcommand{\exercisepart}{\addtocounter{exercisepartnum}{1}\item}


% Macros to make captions print with small type and 'Figure xx' in bold.
\makeatletter
\def\fnum@figure{{\bf Figure \thefigure}}
\def\fnum@table{{\bf Table \thetable}}
\let\@mycaption\caption
%\long\def\@mycaption#1[#2]#3{\addcontentsline{\csname
%  ext@#1\endcsname}{#1}{\protect\numberline{\csname
%  the#1\endcsname}{\ignorespaces #2}}\par
%  \begingroup
%    \@parboxrestore
%    \small
%    \@makecaption{\csname fnum@#1\endcsname}{\ignorespaces #3}\par
%  \endgroup}
%\def\mycaption{\refstepcounter\@captype \@dblarg{\@mycaption\@captype}}
%\makeatother
\let\mycaption\caption
%\newcommand{\figcaption}[1]{\mycaption[]{#1}}

\newcounter{totalcaptions}
\newcounter{totalart}

\newcommand{\figcaption}[1]{\addtocounter{totalcaptions}{1}\caption[]{#1}}

% \psfigures determines what to do for figures:
%       0 means just leave vertical space
%       1 means put a vertical rule and the figure name
%       2 means insert the PostScript version of the figure
%       3 means put the figure name flush left or right
\newcommand{\psfigures}{0}
\newcommand{\spacefigures}{\renewcommand{\psfigures}{0}}
\newcommand{\rulefigures}{\renewcommand{\psfigures}{1}}
\newcommand{\macfigures}{\renewcommand{\psfigures}{2}}
\newcommand{\namefigures}{\renewcommand{\psfigures}{3}}

\newcommand{\figpart}[1]{{\bf (#1)}\nolinebreak[2]\relax}
\newcommand{\figparts}[2]{{\bf (#1)--(#2)}\nolinebreak[2]\relax}


\macfigures     % STATE

% When calling \figspace, make sure to leave a blank line afterward!!
% \widefigspace is for figures that are more than 28pc wide.
\newlength{\halffigspace} \newlength{\wholefigspace}
\newlength{\figruleheight} \newlength{\figgap}
\newcommand{\setfiglengths}{\ifnum\psfigures=1\setlength{\figruleheight}{\hruleheight}\setlength{\figgap}{1em}\else\setlength{\figruleheight}{0pt}\setlength{\figgap}{0em}\fi}
\newcommand{\figspace}[2]{\ifnum\psfigures=0\leavefigspace{#1}\else%
\setfiglengths%
\setlength{\wholefigspace}{#1}\setlength{\halffigspace}{.5\wholefigspace}%
\rule[-\halffigspace]{\figruleheight}{\wholefigspace}\hspace{\figgap}#2\fi}
\newlength{\widefigspacewidth}
% Make \widefigspace put the figure flush right on the text page.
\newcommand{\widefigspace}[2]{
\ifnum\psfigures=0\leavefigspace{#1}\else%
\setfiglengths%
\setlength{\widefigspacewidth}{28pc}%
\addtolength{\widefigspacewidth}{-\figruleheight}%
\setlength{\wholefigspace}{#1}\setlength{\halffigspace}{.5\wholefigspace}%
\makebox[\widefigspacewidth][r]{#2\hspace{\figgap}}\rule[-\halffigspace]{\figruleheight}{\wholefigspace}\fi}
\newcommand{\leavefigspace}[1]{\setlength{\wholefigspace}{#1}\setlength{\halffigspace}{.5\wholefigspace}\rule[-\halffigspace]{0em}{\wholefigspace}}

% Commands for including figures with macpsfig.
% To use these commands, documentstyle ``macpsfig'' must be specified.
\newlength{\macfigfill}
\makeatother
\newlength{\bbx}
\newlength{\bby}
\newcommand{\macfigure}[5]{\addtocounter{totalart}{1}
\ifnum\psfigures=2%
\setlength{\bbx}{#2}\addtolength{\bbx}{#4}%
\setlength{\bby}{#3}\addtolength{\bby}{#5}%
\begin{flushleft}
\ifdim#4>28pc\setlength{\macfigfill}{#4}\addtolength{\macfigfill}{-28pc}\hspace*{-\macfigfill}\fi%
\mbox{\psfig{figure=./#1.ps,%
bbllx=#2,bblly=#3,bburx=\bbx,bbury=\bby}}
\end{flushleft}%
\else\ifdim#4>28pc\widefigspace{#5}{#1}\else\figspace{#5}{#1}\fi\fi}
\makeatletter

\newlength{\savearraycolsep}
\newcommand{\narrowarray}[1]{\setlength{\savearraycolsep}{\arraycolsep}\setlength{\arraycolsep}{#1\arraycolsep}}
\newcommand{\normalarray}{\setlength{\arraycolsep}{\savearraycolsep}}

\newcommand{\hint}{{\em Hint:\ }}

% Macros from /th/u/clr/mac.tex

\newcommand{\set}[1]{\left\{ #1 \right\}}
\newcommand{\abs}[1]{\left| #1\right|}
\newcommand{\card}[1]{\left| #1\right|}
\newcommand{\floor}[1]{\left\lfloor #1 \right\rfloor}
\newcommand{\ceil}[1]{\left\lceil #1 \right\rceil}
\newcommand{\ang}[1]{\ifmmode{\left\langle #1 \right\rangle}
   \else{$\left\langle${#1}$\right\rangle$}\fi}
        % the \if allows use outside mathmode,
        % but will swallow following space there!
\newcommand{\paren}[1]{\left( #1 \right)}
\newcommand{\bracket}[1]{\left[ #1 \right]}
\newcommand{\prob}[1]{\Pr\left\{ #1 \right\}}
\newcommand{\Var}{\mathop{\rm Var}\nolimits}
\newcommand{\expect}[1]{{\rm E}\left[ #1 \right]}
\newcommand{\expectsq}[1]{{\rm E}^2\left[ #1 \right]}
\newcommand{\variance}[1]{{\rm Var}\left[ #1 \right]}
\renewcommand{\choose}[2]{{{#1}\atopwithdelims(){#2}}}
\def\pmod#1{\allowbreak\mkern12mu({\rm mod}\,\,#1)}
\newcommand{\matx}[2]{\left(\begin{array}{*{#1}{c}}#2\end{array}\right)}
\newcommand{\Adj}{\mathop{\rm Adj}\nolimits}

\newtheorem{theorem}{Theorem}
\newtheorem{lemma}[theorem]{Lemma}
\newtheorem{corollary}[theorem]{Corollary}
\newtheorem{xample}{Example}
\newtheorem{definition}{Definition}
\newenvironment{example}{\begin{xample}\rm}{\end{xample}}
\newcommand{\proof}{\noindent{\em Proof.}\hspace{1em}}
\def\squarebox#1{\hbox to #1{\hfill\vbox to #1{\vfill}}}
\newcommand{\qedbox}{\vbox{\hrule\hbox{\vrule\squarebox{.667em}\vrule}\hrule}}
\newcommand{\qed}{\nopagebreak\mbox{}\hfill\qedbox\smallskip}
\newcommand{\eqnref}[1]{(\protect\ref{#1})}

%%\newcommand{\twodots}{\mathinner{\ldotp\ldotp}}
\newcommand{\transpose}{^{\mbox{\scriptsize \sf T}}}
\newcommand{\amortized}[1]{\widehat{#1}}

\newcommand{\punt}[1]{}

%%% command for putting definitions into boldface
% New style for defined terms, as of 2/23/88, redefined by THC.
\newcommand{\defn}[1]{{\boldmath\textit{\textbf{#1}}}}
\newcommand{\defi}[1]{{\textit{\textbf{#1\/}}}}

\newcommand{\red}{\leq_{\rm P}}
\newcommand{\lang}[1]{%
\ifmmode\mathord{\mathcode`-="702D\rm#1\mathcode`\-="2200}\else{\rm#1}\fi}

%\newcommand{\ckt}[1]{\ifmmode\mathord{\mathcode`-="702D\sc #1\mathcode`\-="2200}\else$\mathord{\mathcode`-="702D\sc #1\mathcode`\-="2200}$\fi}
\newcommand{\ckt}[1]{\ifmmode \sc #1\else$\sc #1$\fi}

%% Margin notes - use \notesfalse to turn off notes.
\setlength{\marginparwidth}{0.6in}
\reversemarginpar
\newif\ifnotes
\notestrue
\newcommand{\longnote}[1]{
  \ifnotes
    {\medskip\noindent Note: \marginpar[\hfill$\Longrightarrow$]
      {$\Longleftarrow$}{#1}\medskip}
  \fi}
\newcommand{\note}[1]{
  \ifnotes
    {\marginpar{\tiny \raggedright{#1}}}
  \fi}


\newcommand{\reals}{\mathbbm{R}}
\newcommand{\integers}{\mathbbm{Z}}
\newcommand{\naturals}{\mathbbm{N}}
\newcommand{\rationals}{\mathbbm{Q}}
\newcommand{\complex}{\mathbbm{C}}

\newcommand{\oldreals}{{\bf R}}
\newcommand{\oldintegers}{{\bf Z}}
\newcommand{\oldnaturals}{{\bf N}}
\newcommand{\oldrationals}{{\bf Q}}
\newcommand{\oldcomplex}{{\bf C}}

\newcommand{\w}{\omega}                 %% for fft chapter

\newenvironment{closeitemize}{\begin{list}
{$\bullet$}
{\setlength{\itemsep}{-0.2\baselineskip}
\setlength{\topsep}{0.2\baselineskip}
\setlength{\parskip}{0pt}}}
{\end{list}}

% These are necessary within a {problems} environment in order to restore
% the default separation between bullets and items.
\newenvironment{normalitemize}{\setlength{\labelsep}{0.5em}\begin{itemize}}
                              {\end{itemize}}
\newenvironment{normalenumerate}{\setlength{\labelsep}{0.5em}\begin{enumerate}}
                                {\end{enumerate}}

%\def\eqref#1{Equation~(\ref{eq:#1})}
%\newcommand{\eqref}[1]{Equation (\ref{eq:#1})}
\newcommand{\eqreftwo}[2]{Equations (\ref{eq:#1}) and~(\ref{eq:#2})}
\newcommand{\ineqref}[1]{Inequality~(\ref{ineq:#1})}
\newcommand{\ineqreftwo}[2]{Inequalities (\ref{ineq:#1}) and~(\ref{ineq:#2})}

\newcommand{\figref}[1]{Figure~\ref{fig:#1}}
\newcommand{\figreftwo}[2]{Figures \ref{fig:#1} and~\ref{fig:#2}}

\newcommand{\liref}[1]{line~\ref{li:#1}}
\newcommand{\Liref}[1]{Line~\ref{li:#1}}
\newcommand{\lirefs}[2]{lines \ref{li:#1}--\ref{li:#2}}
\newcommand{\Lirefs}[2]{Lines \ref{li:#1}--\ref{li:#2}}
\newcommand{\lireftwo}[2]{lines \ref{li:#1} and~\ref{li:#2}}
\newcommand{\lirefthree}[3]{lines \ref{li:#1}, \ref{li:#2}, and~\ref{li:#3}}

\newcommand{\lemlabel}[1]{\label{lem:#1}}
\newcommand{\lemref}[1]{Lemma~\ref{lem:#1}}

\newcommand{\exref}[1]{Exercise~\ref{ex:#1}}

\newcommand{\handref}[1]{Handout~\ref{#1}}

\newcommand{\defref}[1]{Definition~\ref{def:#1}}

% (1997.8.16: Victor Luchangco)
% Modified \hlabel to only get date and to use handouts counter for number.
%   New \handout and \handoutwithouttitle commands in newmac.tex use this.
%   The date is referenced by <label>-date.
%   (Retained old definition as \hlabelold.)
%   Defined \hforcelabel to use an argument instead of the handouts counter.

\newcounter{handouts}
\setcounter{handouts}{0}

\newcommand{\hlabel}[2]{%
\stepcounter{handouts}
{\edef\next{\write\@auxout{\string\newlabel{#1}{{\arabic{handouts}}{0}}}}\next}
\write\@auxout{\string\newlabel{#1-date}{{#2}{0}}}
}

\newcommand{\hforcelabel}[3]{%          Does not step handouts counter.
\write\@auxout{\string\newlabel{#1}{{#2}{0}}}
\write\@auxout{\string\newlabel{#1-date}{{#3}{0}}}}


% less ugly underscore
% --juang, 2008 oct 05
\renewcommand{\_}{\vrule height 0 pt depth 0.4 pt width 0.5 em \,}


\setlength{\oddsidemargin}{0pt}
\setlength{\evensidemargin}{0pt}
\setlength{\textwidth}{6.5in}
\setlength{\topmargin}{0in}
\setlength{\textheight}{8.5in}

\newcommand{\theproblemsetnum}{1}
\newcommand{\releasedate}{Tuesday, February 14}
\newcommand{\partaduedate}{Tuesday, February 28}
\newcommand{\tabUnit}{3ex}
\newcommand{\tabT}{\hspace*{\tabUnit}}

\title{6.006 Problem Set 1}

\begin{document}

\handout{Problem Set \theproblemsetnum}{February 14, 2017}

\textbf{All parts are due {\bf \partaduedate} at {\bf 11:59PM}}.

\setlength{\parindent}{0pt}

\medskip

\hrulefill

\medskip

{\bf Name:} Faaya Abate Fulas

\medskip

{\bf Collaborators:} Marwa Abdulhai 

\medskip

\hrulefill

%%%%%%%%%%%%%%%%%%%%%%%%%%%%%%%%%%%%%%%%%%%%%%%%%%%%%
% See below for common and useful latex constructs. %
%%%%%%%%%%%%%%%%%%%%%%%%%%%%%%%%%%%%%%%%%%%%%%%%%%%%%

% Some useful commands:
%$f(x) = \Theta(x)$
%$T(x, y) \leq \log(x) + 2^y + \binom{2n}{n}$
% {\tt code\_function}


% You can create unnumbered lists as follows:
%\begin{itemize}
%    \item First item in a list 
%        \begin{itemize}
%            \item First item in a list 
%                \begin{itemize}
%                    \item First item in a list 
%                    \item Second item in a list 
%                \end{itemize}
%            \item Second item in a list 
%        \end{itemize}
%    \item Second item in a list 
%\end{itemize}

% You can create numbered lists as follows:
%\begin{enumerate}
%    \item First item in a list 
%    \item Second item in a list 
%    \item Third item in a list
%\end{enumerate}

% You can write aligned equations as follows:
%\begin{align} 
%    \begin{split}
%        (x+y)^3 &= (x+y)^2(x+y) \\
%                &= (x^2+2xy+y^2)(x+y) \\
%                &= (x^3+2x^2y+xy^2) + (x^2y+2xy^2+y^3) \\
%                &= x^3+3x^2y+3xy^2+y^3
%    \end{split}                                 
%\end{align}

% You can create grids/matrices as follows:
%\begin{align}
%    A = 
%    \begin{bmatrix}
%        A_{11} & A_{21} \\
%        A_{21} & A_{22}
%    \end{bmatrix}
%\end{align}

\begin{problems}

\section*{Part A}

\problem  % Problem 1


\begin{problemparts}
\problempart \textbf{Group 1:} \\
\\
$ f_{1}(n) = O(n)$\\
$ f_{2}(n) = O(\log(\log n))$\\
$ f_{3}(n) = O(n\log ))$\\
$ f_{4}(n) = O(\log n)$\\
$ f_{5}(n) = n\log \sqrt{n} = 0.5n\log n = O(n\log n)$\\

Since $f_{2}(n)$ reduces the size of the problem by its square root, its order of growth is slower than $f_{4}(n)$  which reduces it by half each time. Therefore, the arrangement of the functions in increasing order of growth is $f_{2}, f_{4}, f_{1}, (f_{3} = f_{5})$ \\



\problempart \textbf{Group2:}\\
\\
$ f_{1} = O(n^{6.006}\log n) $ \\ 
$ f_{2} = n^{2}\log n^{6.006} = 6.006n^{2}\log n =O(n\log n) = O(n^{2}\log n) $\\
$ f_{3} = O(n^{3}) $\\
$ f_{4} = O(n^{2}\log n) $\\
$ f_{5} = O(n^{3}\log n) $\\ 


Arrangement: $ (f_{2} = f_{4}), f_{3}, f_{5}, f_{1}$\\




\end{problemparts}

\problem  I used the Master Theorem method to solve all the recurrences below.% Problem 2

\begin{problemparts}

\problempart $ T(n) = \theta(n)$ since $ n^{\log_{b}a} = \theta(n) > f(n) = \theta(1)  $
\problempart  $ T(n) = \theta(n\lg n )$ since $n^{\log_{b}a} = \theta(n) =  f(n) = \theta(n)$ and $ k=0 $
\problempart $ T(n) = \theta(n^{\lg 3 })$ since $n^{\log_{b}a} = \theta(n^{\lg3}) > f(n) = \theta(n)$
\problempart $ T(n) = \theta(\log n)$ since $n^{\log_{b}a} = \theta(1) < f(n) = \theta(\log n) $
\problempart $ T(n) = \theta(n^{2})$ since $n^{\log_{b}a} = \theta(n) <  f(n) = \theta(n^{2}) $
\problempart $ T(n) = \theta(n^{\lg 7)})$ since $n^{\log_{b}a} = \theta(n^{\lg7}) > f(n) = \theta(n^{2}) $
\end{problemparts}

\problem  % Problem 3

\begin{itemize}
	\item[1. ] Start with $ n= 0 $. Check if $ f(0)\geq 0 $. If so, done. Else, try $n+=1 $.
	\item[2. ]  Check if $ f(1)\geq 0 $. If so, done. Else, try $ n+=1 $.
	\item[3.  ]  Check if $ f(2) \geq $ 0. If so, done. Else, try $n*=2$.
	\item[4. ]  Do step 3 until an $ n $ is found such that $ f(n) \geq 0 $ holds. Then, set $ n $ as an upper bound  and $ \dfrac{n}{2} $ as a lower bound and perform binary search in that interval.

\end{itemize}

	The recurrence for this algorithm would be:\\

 $ T(n) = T(2n) + \O (\log n)$ \\
 
 And, since $n^{\log_{0.5}1} = \theta(1) <  f(n) = \theta(\log n) $, the runtime complexity of the algorithm is  $ \theta(\log n) $. $ n $ can be written as some multiple of $ k $ where $ k \leq n \leq 2k$ holds, so the runtime complexity can also be expressed as $ \theta(\log k) $ 
 

% \begin{problemparts}
% \problempart Part a % Problem 3a
% \problempart Part b % Problem 3b
% \end{problemparts}

\problem  % Problem 4

\textbf{Data Structure:} 
\begin{itemize}
	\item A global  max-heap  of all packages with $ x.priority $ as key
	\item A dictionary(hash table) with $ x.zip$ as key and a local max-heap as value. $ x.priority $ will be the key of the local heaps as well.
	\item Pointers mapping the location of each package in the global heap to its location in the local heap containing it inside the dictionary.
\end{itemize}

Let $ size(D) $ be the size of the global max-heap and $ size(M) $ be the size of some local max-heap in the dictionary. So, $ size(D)  \geq size(M) $ \\

\textbf{Operations}

INIT(Z):

\begin{itemize}
	\item[- ] Initialize an array for the global max-heap and a dictionary for the local max-heaps
\end{itemize}

INSERT(D,x): 
\begin{itemize}
	\item[- ] Invoke the heap operation $ insert(D,x) $ on the global heap. This executes in $O(\log size(D))$ time.
	
	\item[- ] Find $ x.zip  $ in the dictionary of local max-heaps. 
	
	\item[- ] If $ x.zip$ is in the dictionary, invoke $ insert(M,x) $ on the local max heap whose key matches $ x.zip $. This executes in $ O(\log size(M)) $ time. If not, initialize a local max-heap with x as its first element and $ x.zip $ as its key in $ O(1) $ time. 
	
	\item[- ] Overall this would take $O(\log size(D))$ time since that is what dominates all other running times in INSERT(D,x).
\end{itemize}

MOST-URGENT(D): 


\begin{itemize}
	\item[- ] Since the package is a global max, find $ D[0].zip $ in the dictionary and invoke the heap operation $ extract-max(M)  $on the local max-heap paired to it.This executes in $O(\log size(M)) $ time.
	
	\item[- ] Invoke the heap operation $ extract-max(D) $ on the global heap. This executes in $O(\log size(D))$ time.
	
	\item[- ] Overall this would take $O(\log size(D))$ time since that is the costliest running time in MOST-URGENT(D).
\end{itemize}

REGIONAL-MOST-URGENT(D,r): 
\begin{itemize}
	\item[- ] Find $ r $ in the dictionary and locate the maximum priority package M[0] in the global heap using its pointer. This executes in O(1) time because of the use of pointers.

	\item[- ] Invoke the heap operation $ increase-key(D,x,D[0]+1) $ to bubble M[0] up to the top of the heap. This executes in $O(\log size(D))$ time.
	 
	\item[- ] Invoke the heap operation $ extract-max(D) $ to remove it from the heap. This also takes $O(\log size(D))$ time   
	
	\item[- ] Invoke the heap operation $ extract-max(M) $ on the local heap. This executes in $O(\log size(M))$ time.
	
	\item[- ] Overall this would take $O(\log size(D))$ time since that is the costliest running time in REGIONAL-MOST-URGENT(D,r).
	
	
\end{itemize}



\section*{Part B}

\problem
\begin{problemparts}
\problempart \emph{Submit your implementation on alg.csail.mit.edu}
% \problempart \emph{Submit your implementation on alg.csail.mit.edu}
% \problempart \emph{Submit your implementation on alg.csail.mit.edu}
% \problempart Part d % Problem 4d
% \problempart Part e % Problem 4e
\end{problemparts}

\end{problems}

\end{document}

